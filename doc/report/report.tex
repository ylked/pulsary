\documentclass[
  french,
  a4paper,
]{scrartcl}

\usepackage[pages=some]{background}
%\ULCornerWallPaper{1}{arc-template.pdf}
\backgroundsetup{firstpage = true, scale = 1, angle = 0, opacity = 1, contents = {\includegraphics[width = \paperwidth, height = \paperheight] {arc-template.pdf}}}

\usepackage{listings}

\usepackage{xcolor}
\definecolor{codegreen}{rgb}{0,0.6,0}
\definecolor{codegray}{rgb}{0.5,0.5,0.5}
\definecolor{codepurple}{rgb}{0.58,0,0.82}
\definecolor{backcolour}{rgb}{0.98,0.98,0.98}

\lstdefinestyle{mystyle}{
    backgroundcolor=\color{backcolour},   
    commentstyle=\color{codegreen},
    keywordstyle=\color{magenta},
    numberstyle=\tiny\color{codegray},
    stringstyle=\color{codepurple},
    basicstyle=\ttfamily\footnotesize,
    breakatwhitespace=false,         
    breaklines=true,                 
    captionpos=b,                    
    keepspaces=true,                 
    numbers=left,                    
    numbersep=5pt,                  
    showspaces=false,                
    showstringspaces=false,
    showtabs=false,                  
    tabsize=2
}
\lstset{style=mystyle}

\usepackage{amsmath,amssymb}
\usepackage[french]{babel}

\usepackage[T1]{fontenc}
\usepackage[utf8]{inputenc}

\usepackage{lmodern}

\usepackage{xcolor}
\usepackage[margin=2cm,includehead,includefoot]{geometry}
\usepackage{longtable,booktabs,array}
\usepackage{calc}
\usepackage[hidelinks]{hyperref}
\usepackage{etoolbox}

\renewcommand{\arraystretch}{1.2}
\setlength {\parindent}{0em}
\setlength {\parskip}{1em}

\usepackage{fancyhdr}
\pagestyle{fancy}

\title{Pulsary}
\subject{3291.2 JEE/Spring I \(\cdot\) ISC3il-a \( \cdot \) 2023-2024}
\author{Nima Dekhli\\
    \small \href{mailto:nima.dekhli@he-arc.ch}{nima.dekhli@he-arc.ch}}
\date{\today}


\makeatletter
\providecommand{\subtitle}[1]{% add subtitle to \maketitle
  \apptocmd{\@title}{\par {\large #1 \par}}{}{}
}
\makeatother


\begin{document}
\maketitle
\tableofcontents

\section{Introduction}

Pulsary est une API REST qui permet à ses utilisateurs de déterminer 
leur temps de travail. Elle est destinée aux étudiants, aux travailleurs 
indépendants, aux petites entreprises et aux particuliers. Chaque utilisateur 
a la possibilité de créer un compte et de gérer ses chronomètres et ses projets.

Pour chaque tâche qu'il effectue, l'utilisateur peut créer un chronomètre 
qui démarre automatiquement. Une fois la tâche terminée, l'utilisateur 
peut arrêter le chronomètre. Chaque chronomètre peut être associé à un projet, 
ce qui permet de mieux organiser les tâches.

L'API est entièrement réalisée en Java avec le framework Spring. 
Elle utilise une base de données MySQL installée sur Docker pour stocker les données.


\section{Implémentation}

\subsection{Gestion des utilisateurs}

L'application permet la gestion des comptes utilisateurs. 
Chacun peut créer un compte à l'aide d'un nom d'utilisateur et d'un mot de passe. 
Ensuite, il peut se connecter à l'application avec les mêmes informations et 
il obtient une clé d'API. Cette clé est utilisée pour authentifier l'utilisateur
à chaque requête. La clé n'expire pas et est unique pour chaque utilisateur. 
Il est possible de supprimer la clé en se déconnectant. 
Il est possible de changer le nom d'utilisateur et le mot de passe. 

La gestion des accès se fait à l'aide d'un filtre qui vérifie la présence  
et la validité de la clé d'API dans chaque requête (sauf pour l'enregistrement 
du compte et la connexion). Si la clé est invalide, l'utilisateur reçoit une
réponse avec le code d'erreur 401. Si la clé est valide, l'utilisateur est
autorisé à accéder à la ressource demandée, et l'entité utilisateur correspondante 
est ajoutée en temps que variable de requête. Ceci permet d'accéder à l'utilisateur
connecté dans les services. 

\subsection{Gestion des chronomètres}

Les utilisateurs peuvent créer des chronomètres, qui sont automatiquement démarrés 
au moment de leur création. Une fois le travail terminé, l'utilisateur peut arrêter 
le chronomètre. 
Chaque chronomètre possède un nom, une date de début et de fin et un utilisateur. 
De manière optionnelle, il peut avoir un projet qui y est rattaché. 

\subsection{Gestion des projets}

Les utilisateurs peuvent créer des projets qui permettent de regrouper des chronomètres.
Chaque projet possède un nom, une description et un utilisateur. Un projet 
possède également une liste de chronomètres. 

\subsection{Architecture}

L'application utilise une architecture selon trois couches principales : 

\begin{itemize}
    \item La couche Contrôleurs qui permet de répondre aux requêtes
    \item La couche Services qui permet de gérer la logique métier
    \item La couche Données qui permet de gérer les accès à la base de données
\end{itemize}

La règle principale qui est toujours respectée est que chaque couche ne peut
jamais accéder à une couche supérieure. Ceci permet de garantir une bonne
séparation des responsabilités et une bonne modularité.

\subsection{Tests unitaires}

L'application contient des tests unitaires pour toutes les méthodes 
publiques des services et des contrôleurs. Les tests sont réalisés
à l'aide de JUnit 5 et de MockMVC. 


\section{Discussion}

\subsection{Problèmes rencontrés}

La gestion des utilisateurs a été plutôt complexe à mettre en place. 
Il a fallu trouver un moyen de gérer les accès aux ressources de manière 
sécurisée et sans utiliser les sessions, qui n'existent pas dans une API REST. 
La solution a été de permettre aux utilisateurs de générer une clé d'API
sur un endpoint spécifique. Cette clé est ensuite utilisée pour authentifier
l'utilisateur à chaque requête. 


\subsection{Cahier des charges et planification}

Les fonctionnalités décrites dans le cahier des charges n'ont pas 
toutes été implémentées. La gestion des utilisateurs, des chronomètres 
et des projets a été implémentée dans une version simple. Toutefois, 
Il n'y a pas de possibilité d'entrer des données de facturation, 
de récupérer des statistiques ou d'effectuer des opérations de filtrage. 

La planification de ce projet n'a pas été respectée. Alors que selon 
le cahier des charges, le projet aurait dû être développé de manière 
continue tout au long du semestre, il a dans la très grande majorité 
été développé en deux jours. Ceci a inévitablement eu un impact sur
la qualité du code et sur la quantité de fonctionnalités implémentées.

\section{Conclusion}


En résumé, Pulsary est une application performante qui répond 
aux besoins essentiels de gestion du temps pour divers utilisateurs. 
Elle se concentre sur la gestion efficace des chronomètres, des projets, 
et des comptes utilisateurs, avec une authentification sécurisée via une 
clé d'API unique. Bien que certaines fonctionnalités prévues n'aient pas 
été implémentées et que des défis aient été rencontrés, notamment en raison 
d'une planification compressée, Pulsary demeure un outil utile pour le suivi 
du temps de travail. Ce projet souligne l'importance d'une planification 
adéquate et met en évidence les possibilités d'évolution future pour 
enrichir ses fonctionnalités et sa portée.


\end{document}